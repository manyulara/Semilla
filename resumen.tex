\documentclass[10pt,letterpaper]{book}
\usepackage[spanish]{babel}
\usepackage[utf8]{inputenc}
\usepackage{enumitem}


\newenvironment{myepigraph}
  {\par\hfill\itshape
   \begin{tabular}{@{}r@{\hspace{2em}}}} % 2em from the right margin
  {\end{tabular}\par\medskip}
\usepackage[left=2.2cm,right=2.2cm,top=2cm,bottom=2cm]{geometry}
\author{Z. López, M. Lara, E. Ocampo}
\title{Proyecto Educativo Nelson Mandela. Resumen}
\begin{document}

\maketitle
\tableofcontents

\chapter{Introducción}

\begin{myepigraph}A fundamental concern for others\\
in our individual and community lives\\
would go a long way in making the world the better place\\
we so passionately dreamt of.\\
\\
Nelson Mandela
\end{myepigraph}

\section{Presentación}

Somos Zitlali López, Manuel Lara y Eduardo Ocampo. Estamos convencidos que el área en la cual podemos aportar de la forma más eficiente al desarrollo social es la educación. Identificamos en la educación básica un momento clave para el desarrollo ulterior de ciudadanos integrales en el cual podemos incursionar con ideas novedosas, profesionales y basadas en el más profundo deseo de mejorar la situación actual de la sociedad.

Planteamos un proyecto educativo que consiste de una Escuela y un Centro de Investigación Pedagógica (CIP) que funcionarán en la práctica como una misma entidad. Nuestro proyecto estará cimentado en cuatro ejes fundamentales: la {\bf felicidad} basada en la formación integral, vivir la {\bf democracia} participativa, la {\bf disciplina} no autoritaria, que emerge del trabajo naturalmente cuando éste nos motiva y, por último, promover y adoptar una actitud de {\bf exploración} hacia el conocimiento; esto último implica el reconocimiento del carácter dinámico y pragmático del mismo. Nuestro modelo pedagógico es Freinet y estará basado en actividades: buscamos un aprendizaje significativo a través del \emph{hacer}, pues estamos ciertos de que el conocimiento es relevante y pertinente en gran medida por su utilidad. El Centro de Investigación Pedagógica se encargará de generar propuestas educativas concretas (actividades, material didáctico, etc.) siempre congruentes con los ejes de la escuela, actuales, innovadoras y dirigidas a formar comunidad y fortalecer a las familias de sus integrantes. En la escuela se ejecutarán las actividades propuestas, se reportará su efectividad y se elaborarán sugerencias para la mejora de dichas propuestas.
\vspace{1cm}
\begin{myepigraph}No esperes a hacer de tu hijo un buen hombre,\\
hazlo un buen niño.\\Proverbio
\end{myepigraph}

\section{El proyecto}

\subsection{Objetivos}
Son objetivos del proyecto:
\begin{enumerate}
\item Procurar y promover la felicidad de sus integrantes.
\item Proveer de todos los elementos para propiciar el correcto desarrollo emocional de los estudiantes.
\item Fomentar la actitud de exploración. En el caso de los estudiantes, esto significa que proveeremos de todas las herramientas que sean necesarias para que el estudiante se sienta motivado para conocer el mundo que le rodea desde distintas perspectivas. Para los profesores, esto significa una actitud de búsqueda constante de material para relacionar los elementos del temario con el contexto de sus estudiantes. Para los integrantes del CIP, la actualización permanente del temario, estar al tanto de avances en educación y la capacitación constante de los profesores.
\item Impulsar la disciplina en sus integrantes, entendida ésta como el conjunto de métodos que permitan al individuo concretar sus metas idealizadas de la forma más eficiente posible, o en defecto, a adaptarse para proponer nuevas metas que, bajo la perspectiva de la experiencia adquirida, sean factibles y compatibles con el espíritu original que motivó la empresa.
\item La creación de un ambiente democrático, donde todos participen en la construcción de un estado de bienestar general.
\item Buscar la retribución económica justa por las labores realizadas.
\end{enumerate}

\subsection{Centro de Investigación Pedagógica}
El Centro de Investigación Pedagógica (CIP) es una de las propuestas más inovadoras del proyecto, lo diferencian de prácticamente cualquier institución educativa a nivel básico. En el CIP se realizarán investigaciones en educación de calidad internacional y se socializarán los resultados que deriven de ellas a través de la página del proyecto, de publicaciones en revistas arbitradas del área, en congresos que organizará el mismo Centro y congresos externos internacionales y nacionales, así como en un Seminario Permanente del cual serán partícipes los miembros del CIP y los profesores de la Escuela. Realizará además Talleres y Cursos para Profesores interesados. Estamos conscientes de que educar con calidad requiere partir de la experiencia, lo que implica que el entorno dicta los métodos con los que se implementará el temario oficial. Lo anterior no sólo motivará el aprendizaje y favorecerá la trascendencia de los conocimientos adquiridos, sino que representa una herramienta para generar comunidad y fomentar la unión familiar al incentivar intereses comunes entre hijos y padres.\footnote{Las brechas generacionales se agrandan cuando no existen intereses en común y creemos que esto es un elemento importante en la desintegración de las familias.} Los integrantes del CIP deberán, de forma obligatoria, impartir talleres a los estudiantes de la Escuela para conocer de primera mano sus intereses y su situación emocional y académica.
\subsection{Seminario Permanentce}
La conexión fundamental entre el CIP y la Escuela será el Seminario Permanente. Éste será un espacio en el cual se discutirán los métodos de enseñanza que se aplicarán en el salón de clases y aquellas experiencias que fueron positivas en la transmisión del conocimiento y/o en el cumpliimiento de los objetivos del proyecto. Es en el Seminario donde el CIP dará a conocer las propuestas que resulten de su trabajo y el material generado y donde se capacitará a los profesores para que puedan desempeñarse de la mejor manera posible en sus aulas al implementarlos. También en este seminario los profesores darán testimonio del efecto que tienen las implementaciones previamente mencionadas en el salón de clases, se enlistarán las inquietudes que aparecieron frecuentemente por parte de los estudiantes y de los profesores para que el CIP pueda mejorar sus propuestas y/o la capacitación.

\subsection{La Escuela}
Iniciaremos la construcción de una escuela basada en el modelo activo y fundamentada en ideas de pedagogos que promueven los valores que nos identifican y que se plasman en la presentación. La escuela tendrá como principal activo  a sus profesores, quienes además formarán parte del Seminario Permanente de forma obligatoria, de tal forma que su trabajo será de tiempo completo.

Se le dará importancia primordial a la estabilidad emocional de los estudiantes, pues es parte del desarrollo integral y necesario para la el desarrollo de cualquier otra faceta del ser humano.

\subsubsection{Primaria}
En la escuela primaria se realizarán asambleas semanalmente en los salones y mensualmente se convocará a una asamblea de toda la escuela, como lo sugiere el método de Freinet. En el mismo tenor, se llevará un diario por grupo donde se relate lo acontecido en el grupo.

\chapter{Valores del Proyecto}

\begin{myepigraph}
Chi va piano, va sano e lontano.\\
Proverbio
\end{myepigraph}

\section*{Definiciones importantes}
\begin{itemize}

	\item {\bf Comunidad}	
La comunidad existe sólo cuando un conjunto de individuos, independientemente de su número, comparte problemas e intereses.
	
	\item {\bf Problema}
	Un problema es un hecho que usualmente representa alguna dificultad para un individuo o un grupo; esta dificultad donde se obliga a pensar en alternativas para superarla.
\end{itemize}

	\section{Democracia} 

		\subsection{Definición}
La democracia es un proceso de gobierno que se puede desarrollar en cualquier tipo de comunidad (ver definición) para buscar el bien común. Se basa en la capacidad de sus miembros de escuchar y ser escuchados, así como de proponer y actuar -de forma prudente y responsable- en torno a los problemas e intereses que les conciernen.
	\vspace{-0.3cm}
		\subsection{Elementos necesarios en una democracia}	
			\begin{enumerate}[label=\alph*]
			\item La democracia se propone en una comunidad (revisar definición) cuyos individuos tienen interés por los problemas comunes y convencimiento de que se puede encontrar una solución.
			\item Se debe trabajar colectivamente en aproximarse a la solución de los problemas comunes.
			\item Deben encontrarse intereses comunes y establecerse, con base en ellos, objetivos comunes.
			\item Debe existir el interés por generar acuerdos que contemplen la visión de todos los integrantes de la comunidad. En caso de no existir, se debe fomentar el debate sano, entendiendo éste como uno donde se privilegie mejorar los argumentos de las posturas por encima de ``ganar'' el debate. también se deberá inculcar la disposición por ahondar en los intereses de los demás.
			\item Debe existir un ambiente de convivencia, donde se separen los ataques y elogios personales de la argumentación. Se deben poder identificar las falacias argumentativas y quitarles peso en la discusión.
			\item Un modo de proponer donde prevalezca, sobre la idea, los métodos. Sin duda es deseable que quien proponga dirija y asegure el desarrollo de sus propuestas.
			\item Disciplina. Sin esta característica, no se pueden concretar las ideas. Es fundamental. No debe tener la connotación de militarización, si no entenderse como un conjunto de hábitos que permiten a una persona realizar los proyectos que se proponga.
			\end{enumerate}
	\vspace{-0.3cm}	
		\subsection{Sobre el desarrollo de esta característica en los distintos sectores del proyecto (actividades, problemas y notas)}
			\begin{enumerate}[label=\Alph*]
			\item Estudiantes: 
		\begin{itemize}
		\item Asambleas. Se realizarán asambleas grupales y de toda la escuela. En las grupales, se resolverán problemas que surjan en el curso de las sesiones y que no puedan ser resueltos de forma inmediata, o bien que sean recurrentes. Cuando estos no puedan resolverse aún en la asamblea grupal o bien, cuando se trate de un asunto que involucre a gente fuera del grupo, se llevará el asunto a la asamblea de la escuela. En la asamblea general se tratarán entonces estos casos y también los maestros, la dirección y los miembros del CIP buscarán introducir dilemas para que sean resueltos por los estudiantes, como en qué invertir dinero en la escuela, o bien, cuestiones de interacción de la escuela con otras escuelas o comunidades.
		\item Consultas. Los estudiantes serán consultados para conocer su opinión respecto a ciertos aspectos referentes al crecimiento de la escuela; por supuesto, esta opinión será considerada con un fuerte peso para la toma de la decisión.
		\item Gestión de proyectos. Mensualmente, se les dará un prespuesto determinado a cada grupo para que definan algún proyecto y gasten dicho presupuesto.
\end{itemize}					

			\item Profesores: La forma de preparar las clases tendrá un componente democrático importante, pues en el seminario permanente que se desarrollarán muchas ideas como resultado de compartir experiencias y del diálogo constructivo. Una vez consolidada la escuela, se promoverá por parte de los socios que los profesores ganen proporcionalmente con las ganancias de la escuela.
			\item Intendentes y Administrativos: Si es interés de ellos, podrán integrarse a cualquier actividad de la escuela. El CIP promoverá esto.
			\item Miembros del CIP: Deberán estar en contacto continuo con la escuela, dando clases, haciendo labores de intendencia y administrativas, así como tomando cursos y actualizándose continuamente.
			\end{enumerate}

	\section{Felicidad (a través de la formación integral)} 

		\subsection{Algo como una definición:}
		Una actitud hacia la vida que se caracteriza por la ecuanimidad y por la comprensión y la empatía hacia los problemas propios y ajenos.
		\subsection{Profundizando sobre la definición:} 
		\begin{enumerate}[label=\alph*]
		\item Se nutre del reconocimiento de cada una de las facetas del ser (ej. nuestras formas de ser, nuestras emociones y nuestros patrones en distintos contextos: ser apegado, amoroso, enojón, etc.). Este reconocimiento se refleja en la manera en cómo uno responde ante las situaciones externas.
\item La felicidad lleva a buscar la solución de problemas desde la tolerancia (aceptando los desacuerdos y la confrontación, alcanzando la ecuanimidad)  y el trabajo constante.
\item El estado de ánimo de una persona feliz se caracteriza por ser independiente de las condiciones externas inmediatas.
		\end{enumerate}

		\subsection{Sobre el desarrollo de esta característica en los distintos sectores del proyecto (actividades, problemas y notas):}
			\begin{enumerate}[label=\Alph*]
			\item Estudiantes: 
			\begin{itemize}
			\item Los niños deben disfrutar mucho el tiempo que están en la escuela.
			\item Una parte muy importante en la educación y a la que usualmente se le da poca importancia es el estado emocional de nuestros estudiantes. En Semilla los aspectos emocionales son de primera importancia e identificamos que sus soluciones deben de ser previas o simultaneas con el resto del aprendizaje. Resolver conflictos emocionales consideramos que forma parte fundamental de una vida feliz.
\end{itemize}			

			\item Profesores:
			\begin{itemize}
			 \item Queremos que nuestros profesores se den cuenta de lo importante que es su labor. Dignificando su labor seguramente fomentaremos que sean más felices, pues la realización personal está ligada de forma importante a la felicidad.
			 \item Nuestros profesores son de tiempo completo, buscamos darles un sueldo digno para dar a conocer lo importante que son para el proyecto.
			 \item 
\end{itemize}
			\item Intendentes
			\item Administrativos
			\item Miembros del CIP
			\end{enumerate}


	\section{Actitud de exploración} 
	
		\begin{enumerate}[label*=\arabic*.]
		\item Sobre la definición: Explorar no es obtener respuestas, es una actitud y un conjunto de herramientas con las cuales se busca darle sentido al mundo para actuar en él, transformarlo y abordar los problemas que en él vayan surgiendo.
		\item Profundizando sobre la definición:
		\begin{itemize} 
		\item El mundo hace referencia tanto al individuo como a su entorno y a las interacciones entre ambos. 
		\item En este sentido transformar el mundo incluye la transformación del mismo individuo, que usualmente sucede a través de la conformación de intereses.

\item En la exploración tienen lugar la observación, la curiosidad, la creatividad, el análisis (la comparación, la reflexión, la repetición) y la paciencia. Intuición, seguridad.
\item Explorar no es divagar, se realiza en conjunto con la construcción de un método para aumentar las probabilidades de éxito en la solución de los problemas, es decir, la exploración requiere de la disciplina.
\item Explorar no significa cambiar constantemente de intereses, requiere de un cierto compromiso para alcanzar metas.
\item Explorar no significa hacer lo que se desea, la apertura para explorar lo desconocido y lo impuesto (probablemente fuera de lo deseado por desconocido) lleva a la creación de nuevos intereses o a la reafirmación de los intereses existentes.

\end{itemize}
		\item Consideraciones sobre las formas de conocer:
		El acercamiento hacia el conocimiento debe hacerse reconociendo que los seres humanos no enfrentamos algo nuevo con un completo desconocimiento sobre las cosas; los nuevos conceptos se construyen con base en lo que uno ya sabe. Podría suceder que las nuevas ideas no contengan significado para algún individuo, pero para dotarlas de significado el que instruye deberá partir de las estructuras previas en el individuo.
		
		\item Sobre el desarrollo de esta característica en los distintos sectores del proyecto (actividades, problemas y notas)
			\begin{enumerate}[label*=\arabic*.]
			\item Estudiantes: Los porqués tan típicos en los niños de edades alrededor de 3 años, deben ser redirigidos para que el niño pueda formular posibles respuestas. No se debe desalentar el espíritu de búsqueda. Al darse cuenta un individuo de que es capaz de generar repsuestas, tendrá seguridad para ir planteando sus hipótesis y generar modelos del mundo.
			\item Profesores
			\item Intendentes
			\item Administrativos
			\item Miembros del CIP
			\end{enumerate}

		\end{enumerate}
		
	\subsection{Disciplina}
		
		\begin{enumerate}[label*=\arabic*.]
		\item Sobre la definición: Es conjunto de hábitos y habilidades que permiten a un individuo o colectivo alcanzar sus metas; o en su defecto, lograr adaptarlas para, con base en la experiencia adquirida, proponer nuevos objetivos concretos (que contenga el espíritu original) y buscar su realización.
		\item Profundizando en la definición: 
		\begin{itemize}
		\item Una habilidad importantísima es generar el reconocimiento de las propias capacidades para lograr la(s) meta(s); así como evaluar la pertinencia de la perseverancia para lograr la(s) meta(s).
		\item Cuando un individuo o colectivo se plantea sus propias metas es que la disciplina se vuelve fundamental en el desarrollo pleno del individuo o del colectivo.
		\end{itemize}


		\item Sobre el desarrollo de esta característica en los distintos sectores del proyecto (actividades, problemas y notas)
			\begin{enumerate}[label*=\arabic*.]
			\item Estudiantes
			\item Profesores
			\item Intendentes
			\item Administrativos
			\item Miembros del CIP
			\end{enumerate}

		\end{enumerate}



\chapter{Modelo de negocio}

\section{Ejercicio Mensual Financiero Primaria}

\subsection{Gastos}
\begin{center}
\begin{tabular}{cccc}
Concepto & Costo Unitario & Cantidad & Gasto \\ 
Profesor & 15000 & 6 & 90,000 \\ 
Pago mensual de Renta & 25,000 & 1 & 25,000 \\ 
Dirección & 20000 & 1 & 20,000 \\
Papelería & 5000 & 1 & 5,000 \\
CIP & 15000 & 3 & 45,000 \\
Total &  &  & 200,0000 \\
\end{tabular} 
\end{center}

\subsection{Ingresos}
Suponiendo una colegiatura de 4000 por estudiante, tenemos:\\
\begin{center}
\begin{tabular}{cccc}

Concepto & Ingreso por estudiante & Cantidad & Ingreso neto \\ 

Colegiaturas & 4000 & 90 & 360,000 \\ 

\end{tabular} 
\end{center}

\subsection{Balance}

\chapter{Colaboraciones}

\chapter{Anexos}
	\section{Sobre la forma de publicitar} 
	\subsection{Democracia} Es importante remarcar que la democracia es una palabra pervertida. Para buscar que la gente entienda lo que queremos decir con democracia, debemos ser sutiles y evitar el uso de esa palabra.
	\subsection{Felicidad} Existen papás que podrían dudar sobre 

\end{document}
